\pagenumbering{Roman} 
\begin{center}
\textbf{RESUMEN}
\end{center}

El objetivo principal de la tesis se justifica por el problema de las altas horas improductivas que se pasan en reuniones con casi ningún marco conceptual para enfrentar este gran dilema que es ubicuo en el mundo. Luego,  se busca un \textit{software} donde cada participante de la reunión genere la colaboración y coordinación de actividades mediante los elementos de diálogo: duda, desacuerdo, compromiso, norma, acuerdo y compromiso individual de forma tal que facilite el retomar el estado de reuniones pasadas. 

El documento tiene sus cimientos en la teoría del diálogo/acción, donde el diálogo se entiende como una forma de comunicación que facilita la creación. Para este caso se estudian los diferentes \textit{meetingware} que apoyan la gestión, haciendo uso de elementos de comunicación articulado con la acción.

Posteriormente, se plantea una de idea de innovación aplicando un benchmarking a diferentes herramientas del mercado que permiten generar actas de reuniones, de las cuales muy pocas utilizan elementos del diálogo para el seguimiento de tareas. Lo anterior permitió generar un modelo liviano de negocio mediante metodología Lean Startup.

En consecuencia, se crea un \textit{software} denominado D-Minute, que incorpora la teoría del diálogo al momento de la confección de un acta de reuniones, donde los participantes interactúan en el mismo espacio de manera síncrona. En la creación del producto se utilizó Scrum como metodología de desarrollo de \textit{software}, dentro de este marco de trabajo se definen las épicas que dan origen release map del producto. Se utilizó la arquitectura de Netflix OSS para el \textit{backend} y Angular 5 para el \textit{frontend}. 

Se concluye por el lado de la innovación un Lienzo Canvas para el modelo de negocio del mínimo producto viable, por el lado de la investigación el diseño del experimento que debe ser llevado a cabo para validar las hipótesis que dieron origen al \textit{software} D-Minute y por último una base para trabajos futuros.\newline 
\newline\textbf{Palabras Clave:} Trabajo Colaborativo, Diálogo, \textit{Meetingware}