\chapter*{Resumen} % si no queremos que añada la palabra "Capitulo"
\addcontentsline{toc}{chapter}{Resumen} % si queremos que aparezca en el índice
El objetivo principal de la tesis es generar un tipo de actas de reuniones que facilite recuperar el estado de reuniones pasadas. Para lograr esto se presenta - en el mismo documento - la creación de un software denominado D-Minute, que incorpora la teoría del diálogo al momento de la confección de un acta de reuniones donde los participantes interactúan en el mismo espacio de manera síncrona. El tema se justifica por el problema de las altas horas improductivas que se pasan en reuniones con casi ningún marco conceptual para enfrentar este gran dilema que es ubicuo en el mundo. Luego,  se busca un software donde cada participante de la reunión genere la colaboración y coordinación de actividades mediante los elementos de dialogo: duda, desacuerdo, compromiso, norma, acuerdo y compromiso individual. El documento tiene sus cimientos en la teoría del diálogo/acción, donde el diálogo se entiende como una forma de comunicación que facilita la creación y este caso un meetingware que apoya la gestión haciendo uso elementos de comunicación articulado con la acción.
Este trabajo también dio lugar a una investigación de innovación, donde se analizaron diferentes herramientas del mercado que permiten generar actas de reuniones de las cuales muy pocas utilizan elementos del diálogo para el seguimiento de tareas. Lo anterior permitió generar un modelo liviano de negocio mediante metodología Lean Startup.
Una parte relevante de la tesis fue la utilización de Scrum como metodología de desarrollo de software, pasando las épicas a su correspondiente release map y las historias de usuario asociadas. La segunda parte relevante fue el desarrollo en microservicios utilizando la arquitectura de Netflix OSS combinado con angular 5 en la parte frontal.
Por último, se realiza un benchmarking en base a los criterios que debe poseer una herramienta de gestión de reuniones, los criterios fueron tomados de la revisión literatura y del análisis de los sistemas de mercado. El resultado final es la medición de las características del software D-Minute con sus competidores de mercado y una base para trabajos futuros. \newline 
\newline\textbf{Palabras Clave:} Trabajo Colaborativo, Diálogo, Meetingware