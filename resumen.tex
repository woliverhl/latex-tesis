\pagenumbering{Roman} 
\begin{center}

\textbf{\large{RESUMEN}}

\end{center}

El objetivo principal de la tesis, se justifica por el problema de las altas horas improductivas que se pasan en reuniones con casi ningún marco conceptual para enfrentar este gran dilema que es ubicuo en el mundo. Se propone una minuta electrónica, llamada D-Minute, para sostener reuniones eficaces donde cada participante de la reunión genere la colaboración y coordinación de actividades mediante los elementos de diálogo: duda, desacuerdo, norma, acuerdo y compromiso individual de forma tal que facilite el retomar el estado de reuniones pasadas. 

La tesis adopta el enfoque diálogo/acción para entender las reuniones en las cuales se articula el lenguaje con la acción en el proyecto mismo. Además, este enfoque caracteriza tantos los momentos de convergencia como los momentos de divergencia y los relaciona con tareas derivadas del acto de reunirse.  Se postula que el registro de estos momentos y sus elementos asociados, facilita el retomar el estado de reuniones pasadas de manera eficaz y expedita.

La metodología para desarrollar la tesis tiene tres fuentes I+D+i. En la práctica, la investigación entrega un diseño experimental para validar la eficacia de D-Minute con un experimento científico con dos equipos de proyectos - uno de control y otro experimental - para alegar mejoras basados en los tiempos de respuesta y circulación del conocimiento. La parte de innovación entrega un \textit{benchmark} con \textit{meetingware} existentes para justificar que D-Minute es un producto mínimo viable, con valor para un potencial segmento de clientes. De hecho, usando \textit{Lean} Canvas se pudo generar un modelo liviano de negocios usando la metodología \textit{Running Lean} que pone de relieve este hecho. Finalmente, la parte D es el \textit{meetingware} mismo D-Minute. En la creación del producto se utilizó: \textit{Scrum} como metodología de desarrollo de \textit{software}, Netflix OSS como arquitectura de \textit{Back-End}, Angular 5  como \textit{framework} para \textit{Front-End} y \textit{Docker} para el despliegue de micro servicios.\\

\begin{flushleft}
\textbf{Palabras Clave:} Trabajo Colaborativo, Diálogo, \textit{Meetingware}
\end{flushleft}