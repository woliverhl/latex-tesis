\section{Encuesta KNA}

Checklist “Evaluación de Aplicaciones Colaborativas desde la aproximación de la Administración del Conocimiento”\newline

Basado en el artículo: Evaluating Collaborative Applications from a Knowledge Management Approach, Aurora Vizcaíno, Manuel Martínez, Gabriela Aranda, Mario Piattini, University of Castilla-La Mancha, Escuela Superior de Informática, España.\newline

Encuesta de Calidad para la Herramienta: \underline{            }

\begin{enumerate}[1.]
    \item Creación de conocimiento
    
\begin{table}[!h]
\centering
\resizebox{15cm}{!} {
\begin{tabular}{|l|r|l|r|l|l|}
\hline
\multicolumn{1}{|c|}{\textbf{}} & \multicolumn{1}{c|}{\textbf{Nunca}} & \multicolumn{1}{c|}{\textbf{Rara Vez}} & \multicolumn{1}{c|}{\textbf{A Veces}} & \textbf{Frecuentemente} & \multicolumn{1}{c|}{\textbf{Siempre}} \\ \hline
\begin{tabular}[c]{@{}l@{}}a.- ¿La herramienta ayuda a encontrar la información que busca?, \\ ¿El sistema ayuda a entenderla?,  \\ Por ejemplo, ¿el sistema muestra ejemplos para aclarar los conceptos?\end{tabular} &  &  &  &  &  \\ \hline
b.- ¿El sistema propone soluciones a los problemas? &  &  &  &  &  \\ \hline
c.- ¿Tiene la herramienta algún mecanismo para explicar las soluciones que muestra? &  &  &  &  &  \\ \hline
\end{tabular}
}
\end{table}    
    
    \item Acumulación de conocimiento
    
\begin{table}[!h]
\centering
\resizebox{15cm}{!} {
\begin{tabular}{|l|r|l|r|l|l|}
\hline
\multicolumn{1}{|c|}{\textbf{}} & \multicolumn{1}{c|}{\textbf{Nunca}} & \multicolumn{1}{c|}{\textbf{Rara Vez}} & \multicolumn{1}{c|}{\textbf{A Veces}} & \textbf{Frecuentemente} & \multicolumn{1}{c|}{\textbf{Siempre}} \\ \hline
\begin{tabular}[c]{@{}l@{}}a.-¿ La herramienta tiene un repositorio donde la información se pueda almacenar?,\\ ¿Esta base de datos tiene suficiente calidad?\end{tabular} &  &  &  &  &  \\ \hline
b.- ¿La herramienta tiene mecanismos inteligentes para capturar la información? &  &  &  &  &  \\ \hline
c.- ¿La herramienta ayuda a documentar las actividades diarias? &  &  &  &  &  \\ \hline
\end{tabular}
}
\end{table}     
    
    \item Compartir conocimiento
    
\begin{table}[!h]
\centering
\resizebox{15cm}{!} {
\begin{tabular}{|l|l|l|l|l|l|}
\hline
\multicolumn{1}{|c|}{\textbf{}} & \multicolumn{1}{c|}{\textbf{Nunca}} & \multicolumn{1}{c|}{\textbf{Rara Vez}} & \multicolumn{1}{c|}{\textbf{A Veces}} & \textbf{Frecuentemente} & \multicolumn{1}{c|}{\textbf{Siempre}} \\ \hline
\begin{tabular}[c]{@{}l@{}}a.- ¿La herramienta tiene mecanismos para comunicarse con otras personas?, \\ ¿Qué mecanismos tiene: síncrono (chat, videoconferencia, pizarras compartidas), \\ asíncrono (email, listas de mail, grupos de noticias, foros asíncronos)?\end{tabular} & \multicolumn{1}{r|}{} &  & \multicolumn{1}{r|}{} &  &  \\ \hline
b.- ¿La herramienta tiene mecanismos para ubicar a expertos? & \multicolumn{1}{r|}{} &  & \multicolumn{1}{r|}{} &  &  \\ \hline
c.- ¿Tiene la herramienta algún mecanismo para explicar las soluciones que muestra? & \multicolumn{1}{r|}{} &  & \multicolumn{1}{r|}{} &  &  \\ \hline
d.- ¿La herramienta tiene mecanismos para rastrear el trabajo de estas comunidades de práctica? &  &  &  &  &  \\ \hline
\begin{tabular}[c]{@{}l@{}}e.- ¿El sistema tiene un mecanismo para enviar nueva información a aquellos \\ miembros del equipo que podrían necesitarla?\end{tabular} &  &  &  &  &  \\ \hline
f.- ¿La herramienta tiene control de flujos de trabajo? &  &  &  &  &  \\ \hline
\begin{tabular}[c]{@{}l@{}}g.- ¿La herramienta tiene mecanismos para detectar que persona puede saber sobre un tema \\ o información que otra persona necesite?\end{tabular} &  &  &  &  &  \\ \hline
\end{tabular}
}
\end{table}    
    
    \item Utilización de conocimiento
    
\begin{table}[!h]
\centering
\resizebox{15cm}{!} {
\begin{tabular}{|l|r|l|r|l|l|}
\hline
\multicolumn{1}{|c|}{\textbf{}} & \multicolumn{1}{c|}{\textbf{Nunca}} & \multicolumn{1}{c|}{\textbf{Rara Vez}} & \multicolumn{1}{c|}{\textbf{A Veces}} & \textbf{Frecuentemente} & \multicolumn{1}{c|}{\textbf{Siempre}} \\ \hline
a.- El sistema tiene métodos para buscar o filtrar información? &  &  &  &  &  \\ \hline
b.- ¿El sistema tiene técnicas para evitar ruido en la información? &  &  &  &  &  \\ \hline
\begin{tabular}[c]{@{}l@{}}c.- ¿El sistema tiene mecanismos de alerta para aconsejar al usuario para \\ consultar información o contactar a otras personas?\end{tabular} &  &  &  &  &  \\ \hline
d.- ¿El sistema recomienda la mejor forma de llevar a cabo una tarea? & \multicolumn{1}{l|}{} &  & \multicolumn{1}{l|}{} &  &  \\ \hline
\end{tabular}
}
\end{table}      
    
    \item Internalización del Conocimiento
    
\begin{table}[!h]
\centering
\resizebox{15cm}{!} {
\begin{tabular}{|l|r|l|r|l|l|}
\hline
\multicolumn{1}{|c|}{\textbf{}} & \multicolumn{1}{c|}{\textbf{Nunca}} & \multicolumn{1}{c|}{\textbf{Rara Vez}} & \multicolumn{1}{c|}{\textbf{A Veces}} & \textbf{Frecuentemente} & \multicolumn{1}{c|}{\textbf{Siempre}} \\ \hline
a.- ¿El sistema ayuda a aprender cómo llevar a cabo una tarea? &  &  &  &  &  \\ \hline
b.- ¿El sistema tiene un mecanismo para mejorar las habilidades del empleado? &  &  &  &  &  \\ \hline
\begin{tabular}[c]{@{}l@{}}c.- ¿El sistema tiene técnicas para enseñar filosofía organizacional, estándares,\\  y perfiles de clientes?\end{tabular} &  &  &  &  &  \\ \hline
\end{tabular}
}
\end{table}     
    
    \item Integración de conocimiento
    
\begin{table}[!h]
\centering
\resizebox{15cm}{!} {
\begin{tabular}{|l|r|l|r|l|l|}
\hline
\multicolumn{1}{|c|}{\textbf{}} & \multicolumn{1}{c|}{\textbf{Nunca}} & \multicolumn{1}{c|}{\textbf{Rara Vez}} & \multicolumn{1}{c|}{\textbf{A Veces}} & \textbf{Frecuentemente} & \multicolumn{1}{c|}{\textbf{Siempre}} \\ \hline
a.- ¿El sistema facilita la integración de información y conocimiento? &  &  &  &  &  \\ \hline
\end{tabular}
}
\end{table}        
\end{enumerate}