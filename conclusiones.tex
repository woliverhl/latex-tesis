\section{CONCLUSIONES}

El siguiente capítulo expone las conclusiones de este trabajo de tesis - cuyo mínimo producto viable fue D-Minute - pasando por la investigación, el desarrollo y una pequeña parte de innovación I+D+i. De lo anterior se puede indicar que se va recapitular los contenidos del capítulo primero para concluir en base los objetivos planteados; como fueron realizados y en qué forma se llevaron a cabo. Luego, se proponen los alcances producto de la ejecución para indicar los trabajos futuros que se deben seguir con el objeto de avanzar en el estudio. Finalmente se presenta una reflexión personal acerca del trabajo.


\subsection{OBJETIVOS}

En la sección 1.4.2 de este documento se presentan los objetivos específicos que deben ser llevados a cabo para lograr el objetivo general del proyecto. En base a estos se elabora las siguientes conclusiones de cada unos de ellos:

\subsubsection{Objetivo específicos}

\textit{OBJETIVO 1:  “Hacer una revisión de la literatura de los \textit{meetingware} con sus características para relacionarlo con D-Minute"}\newline

Se logra el objetivo planteado el cual puede verse en detalle en el capítulo dos del mismo documento así como sus correspondientes fuentes bibliográficas. Se indican algunos de los principales \textit{software} de reuniones que poseen una base científica y de lo anterior se logra establecer el marco de investigación.\newline

\textit{OBJETIVO 2:  “Comparar D-Minute con otros \textit{meetingware} en términos comerciales"}\newline

Se logra el objetivo planteado el cual puede verse en detalle en el capítulo tres, se exponen diferentes \textit{meetingare} comerciales algunos de ellos muy exitosos. De esta información se genera un \textit{benchmarking} del producto, realizando una comparación con los criterios de evaluación detectados tanto en la revisión bibliográfica como en la revisión de mercado sobre lo que es necesario en un \textit{software} de reuniones.\newline

\textit{OBJETIVO 3:  “Dise\~nar una propuesta de valor de D-Minute cómo \textit{meetingware} tendiente a convertirse en una potencial innovación en el mercado"}\newline

Se logra el objetivo planteado el cual puede verse en detalle en el capítulo tres donde se expone el lienzo de canvas D-Minute desarrollado con metodología Lean Startup. La formulación de la idea nace de la revisión de mercado expuesta en el mismo capítulo cuyo primer objetivo es entregar una herramienta al mercado que posea una síntesis dialógica como pilar para la recuperación del contexto como medio para tener reuniones efectivas.\newline

\textit{OBJETIVO 4:  “Desarrollar el artefacto de \textit{software} D-Minute con el principio de las actas dialógicas que permita administrar reuniones con los elementos del diálogo y las tareas asociadas"}\newline

Se logra el objetivo planteado el cual puede verse en detalle en el capítulo cuatro. El desarrollo del \textit{software} tuvo dos enfoques: en primer lugar conocer la opinión  - mediantes encuestas realizadas - de diferentes gestores de proyecto de empresas como Banco BCI, Caja Los Andes, Comder y Banco de Chile sobre los criterios que debe incluir un sistema de este tipo. En segundo lugar la incorporación de la síntesis dialógica en herramientas de reuniones, no solamente el compromiso sino el resto de elementos (dudas, desacuerdos, acuerdo, norma común, acuerdos de coordinación y tarea) que se pasan por alto en estas instancias.\newline

\textit{OBJETIVO 5:  “Dise\~nar una validación de D-Minute por medio de una investigación científica para mostrar las ventajas en tiempos y en circulación del conocimiento2}\newline

Se logra el objetivo planteado el cual puede verse expuesto en el capítulo cinco. Se presenta el diseño del experimento, como debe ser ejecutado, la recolección de datos y lo necesario para que el experimento puede ser ejecutado en otro estudio.

\subsubsection{Objetivo general}

\textit{OBJETIVO GENERAL: “Desarrollar un tipo actas de reuniones que facilite la recuperación, el estado de reuniones pasadas y el flujo del conocimiento basada en el enfoque diálogo/acción"}\newline

En base a los objetivos específicos cumplidos en su mayoría y apoyada por la información expuesta en este mismo documento, se ha cumplido el objetivo general de la tesis. D-Minute es un \textit{software} del área de los meetingware que fue desarrollado con tecnología de vanguardia, se aloja en un servidor PaaS y posee la teoría del diálogo para el seguimiento de reuniones. 

\subsection{CONCLUSIONES GENERALES}

El desarrollo de esta tesis estuvo ligada tanto a la investigación como al desarrollo, pero sin embargo pudimos incorporar una linea mas pequeña. La innovación, que sin duda aporta un tremendo valor cuando se combina con investigación y desarrollo pues podemos ver como una idea empieza a tomar forma al punto de contribuir a la sociedad.

La investigación realizada expuso la diversidad de sistemas que existen en el área de meetingware y como cada uno de ellos ha contribuido al apoyo de las personas a lo largo de los años, debido a que las reuniones son la forma de coordinar el trabajo colaborativo. Además, el desarrollo de la aplicación fue utilizando la tendencia del mercado, el uso de micro arquitectura para la creación de \textit{software} permite mayor velocidad en los desarrollos, escalabilidad y \textit{performance} que mezclada con PaaS logra una alta disponibilidad del producto.

Se emplearon dos metodologías para la creación del producto como del modelo de negocio. Scrum para el desarrollo de \textit{software} que permitió lograr el MVP pues en cada iteración se dio foco a las principales funciones y no al todo del producto. Por otra parte, para el desarrollo del modelo liviano de negocio se utilizó Lean Startup que permitió descubrir a qué mercado estaría enfocado el producto como los clientes finales del mismo.

Finalmente y dado que los objetivos se dan por cumplidos, durante la revisión bibliográficas se presentaron dos preguntas de investigación de mercado a responder:

\begin{enumerate}[1.]
	\item ¿D-Minute expone los elementos de diálogo mejor que otras herramientas síncronas?

R: En base al cuadro comparativo realizado en el capítulo tres, D-Minute expone todos los elementos de diálogo, genera una trazabilidad de los elementos y permite un seguimiento de ellos.

	\item ¿Qué ventaja competitiva posee D-Minute respecto a los productos comerciales existentes hoy día en el mercado?

R: Además de los diferentes elementos de diálogo que emplea D-Minute, a diferencia de sus competidores que solo utilizan los compromisos la gestión del proyecto. D-Minute ofrece un seguimiento de tareas para los compromisos y es una plataforma que opera bajo el concepto de simple en términos de usabilidad.

\end{enumerate}

\subsection{ALCANCES}

Las conclusiones del trabajo realizado se aplican principalmente al desarrollo de una aplicación y su respectivo diseño comparativo. Por tanto se espera que lo observado en el documento sea generalizable a meetingware que utilicen la síntesis dialógica dentro de sus bases. 

\subsection{TRABAJOS FUTUROS}

Si bien el trabajo cumple con los objetivos planteados en el tiempo disponible para ser ejecutado. Deja muchos espacios de mejora que pueden ser abordados en futuros trabajos para concluir la investigación:

\begin{enumerate}[1.]
	\item Capacidad de generar un complemento para obtener la agenda del gestor. Esto sería de gran valor pues habría información pre-cargada como por ejemplo: fecha y hora de reunión, asistentes y descripción, lo cual permitiría a cada usuario completar la información en vez de crearla en dos sistemas.
	\item Generar \textit{login} con una cuenta de redes sociales u alguna corporativa para evitar múltiples claves de sistemas.
	\item Capacidad de adjuntar documentos relevantes tanto en la reunión ejecutada como en el proyecto con el fin de disponer información a los participantes.
	\item Generar recordatorios de las tareas asociadas a los compromisos adquiridos en las reuniones utilizando la fecha de entrega y el estado como parámetro de seguimiento. 
	\item Generar una versión \textit{responsive} para IOS y Android pues hoy en día la tendencia es a utilizar celulares para el seguimiento de temas. Lo que podría permitir mayor adaptación de la herramienta como medio para reuniones efectivas.
	\item Probar científicamente que D-Minute permite recuperar el contexto de reuniones pasadas utilizando la teoría del diálogo. 
	\item Publicar los resultados del punto seis en un artículo científico para generar nuevas investigaciones de la teoría del diálogo en reuniones de trabajo.

\end{enumerate}

\subsection{REFLEXION PERSONAL}

Este trabajo se ideó como una forma de contribuir en la línea de \textit{meetingware}; sin embargo y personalmente dejo mucho mas que eso, pues siempre queremos enfocar hacia donde deben ir las personas con nuestros sistemas pero al revés es algo distinto. Desde una forma de comunicación diseñar un software que permita las personas usar el recurso más valioso que hoy tenemos es sin duda un trabajo gigantesco. El tiempo es lo que más cuidamos y estar en reuniones que no aportan valor, pues al menos la mitad del tiempo estamos entregando contexto de lo revisado anteriormente, genera un desgaste en diversos aspectos.

La teoría del diálogo es algo nuevo tanto para el alumno como para muchas personas, debemos tender a una nueva forma de abordar los problemas a través de la comunicación  efectiva. Que en un contexto de reuniones pasar a ser muy efectiva si lo que buscamos es que la otra persona me entienda. Por estas razones este trabajo no solo trajo consigo la elaboración de un documento sino descubrir una forma emplear el diálogo en contextos de conversación.

Finalmente, los trabajos de investigación estuvieron muy ligados al desarrollo de nuevas tendencias en la industria del \textit{software}, las nuevas tecnologías tomadas en este trabajo dejan una base potente para los trabajos futuros que desean aportar conocimiento sobre la línea de meetingware u otra.




